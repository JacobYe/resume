% !TEX TS-program = xelatex
% !TEX encoding = UTF-8 Unicode
% !Mode:: "TeX:UTF-8"

\documentclass[12pt,a4paper]{resume}
\begin{document}
\XeTeXlinebreaklocale "zh"

%%%%%%%%%%
%
% HEADER
%
%%%%%%%%%%
\namesection{Junjie Ye}{(Jacob)}

\vspace{3pt}
\subtitle{1991, Male, Shanghai}
\vspace{3pt}
\subtitle{TEL: 15867835270}
\vspace{3pt}
\subtitle{MAIL: junjieyemain@outlook.com}

\indent\makebox[\linewidth]{\rule{\paperwidth}{0.3pt}}



%%%%%%%%%%
%
% COLUMN ONE
%
%%%%%%%%%%
\begin{minipage}[t]{0.32\textwidth} 

%%%%%%%%%%
% EDUCATION
\section{Education Background}

\subsection{University of Edinburgh}
\subtitle{MSc \hfill 2014/09 - 2015/11}
\desc{
	Artificial Intelligence \hfill \\
	%Learning From Data Specialism \\
	%MSc in Artificial Intelligence \\
	GPA: 65
}
\sectionsep

\subsection{Zhejiang University}
\subsection{of Finance \& Ecomomics}
\subtitle{Dual Degree \hfill 2010/09 - 2014/06}
\desc{
	Computer Science \\
	GPA: 88/100 \\
	Taxation \\
	GPA: 89/100
}
\sectionsep

%%%%%%%%%%
% PROFILE
\section{Profile}
LinkedIn:  \href{https://www.linkedin.com/in/junjie-ye-49477b126/}{\custombold{Junjie Ye}}

\sectionsep

%%%%%%%%%%
% SKILL
\section{Skills}

\subsection{Certificate}
\vspace{13pt}
%\vspace{\topsep}
\begin{lists} \desc {
	\item Qualified Software Engineer
	\item Stanford University \\Machine Learning Certificate
}
\end{lists}

\subsection{Domain}
\vspace{13pt}
\begin{lists} \desc{
	\item Machine Learning / Deep Learning
	\item Probabilistic Modelling / Bayesian Inference
	\item Reinforcement Learning / Game Theory
	\item Natural Language Process
	\item Speech Recognition
	\item Knowledge Graph
	\item Big Data / Data Mining
	\item Crawler
}
\end{lists}

\subsection{Projects}
\vspace{13pt}
\begin{lists} \desc{
	\item HMM-GMM ASR System
	\item Kaggle Car Driver Identification
	\item Reinforcement Learning Coursework
	\item Game Theory Coursework
	\item Information Theory Coursework
	\item Business Intelligence
	\item Book Management System
	\item Tax Invoicing System
}
\end{lists}
%C \textbullet{} C++ \textbullet{} Assembly \textbullet{} \LaTeX \textbullet{} Markdown \\
\sectionsep

\subsection{Skillset}
\vspace{13pt}
\begin{lists} \desc{
	\item Tensorflow / Theano / Keras
	\item Hadoop / Spark / Hbase
	\item Docker
	\item Scrapy
	\item Flask
	\item MongoDB
	\item Neo4j / ArangoDB
	\item Redis
	\item Weka
}
\end{lists}

\subsection{Programming Language}
%\begin{lists} \desc{
%	\item Python
%	\item Java
%	\item Shell
%	\item Matlab
%	\item C / C++
%	\item SQL
%	\item JavaScript
%	\item HTML / CSS
%	\item Markdown
%	\item \LaTeX
%}
%\end{lists}
\desc{
Python \textbullet{} Java \textbullet{} Shell \textbullet{} Matlab \textbullet{} C/C++ \textbullet{} SQL \textbullet{} JavaScript \textbullet{} HTML/CSS \textbullet{} Markdown \textbullet{} \LaTeX
\sectionsep
}


%%%%%%%%%%
%
% COLUMN TWO
%
%%%%%%%%%%
\end{minipage} 
\hfill
\begin{minipage}[t]{0.65\textwidth} 

\section{Working Experience}

\subsection{Emotibot Technologies Limited \hfill Shanghai, China}
\subtitle{Research Engineer \hfill 2016/04 - Now}
\vspace{12pt}
\begin{lists} \desc{
	\item Bot Memory Team: \\Design and implement Multi-Modal Architecture (Rule Engine + Machine Learning / Deep Learning) to memorize user information, including personal information, emotional status, habbits etc., within a conversational UI. Thereby making the bot more intelligent in understanding the context and intents. \\The system is implememted by Java, intergrated with multiple learning models via RESTful API. The deep learning part is based on TensorFlow. The input of the models is transformed by word2vec from raw sentences to vectors. The learning models use various architectures such as RNN, LSTM, GNU. The online accuracy of the system is over 90\% and the offline accuracy is over 80\%. The emotional status module is applying for a patent.
	\item Knowledge Graph Team: \\Construct large-scale knowledge graph, including general knowledge and domain knowledge (music, movies, sport, e-commerce etc.). Use Scrapy to build the web crawler part, and store data into HBase / MongoDB. Run ETL via Map-Reduce and data-mining via Spark. Store the knowledge entities and relations by Graph DBs (Neo4j / ArangoDB). The current scale of the knowledge graph has exceeded 10 million entities and millions of relations.
}
\end{lists}
\sectionsep

\subsection{Standard Chartered Bank \hfill Ningo, China}
\subtitle{IT Department Intern \hfill 2013/10月 - 2014/01}
\begin{lists} \desc{
	\item Co-operate credit evaluation for enterprises. Following and reporting customer credit evaluation results.
	\item Co-operate bank’s financial systems maintenance.
}
\end{lists}
\sectionsep


\subsection{Ningbo Economic \& Information Commission \hfill Ningbo, China}
\subtitle{Intern \hfill 2013/01 - 2013/02}
\begin{lists} \desc{
	\item Learn industrial policies and standards of IT industry.
}
\end{lists}
\sectionsep

\subsection{Xerox Businiess Co., Ltd. \hfill Ningbo, China}
\subtitle{IT Department and Dispatch Center Intern \hfill 2012/07 - 2012/08}
\begin{lists} \desc{
	\item Deal with computer maintenance and software fault.
}
\end{lists}
\sectionsep

\subsection{China Telecom \hfill Ningbo, China}
\subtitle{Network Departmant Intern \hfill 2011/07 - 2011/08}
\begin{lists} \desc{
	\item Co-operate network breakdown disposal.
}
\end{lists}
\sectionsep


%%%%%%%%%%
% RESEARCH
\section{Research Experience}

\subsection{Recurrent Neural Network Acoustic Models}
\subtitle{Postgraduate Thesis \hfill 2015/06 - 2015/08}
\desc{Worked with \textbf{\href{http://homepages.inf.ed.ac.uk/srenals/}{Prof Steve Renals}} and \textbf{\href{http://homepages.inf.ed.ac.uk/llu/}{Liang Lu}} in using recurrent neural network to address acoustic modelling problem in the Speech Recognition systems. The goal is to achieve better performance by recurrent neural network acoustic models over deep neural network acoustic models, and to investigating methods to improve recurrent neural network acoustic models. The project used Theano and Kaldi for constructing the whole Speech Recognition system.}
\sectionsep

%\subsection{车辆保险用户预测模型}
%\subtitle{数据挖掘项目 \hfill 2015年3月 - 2015年5月}
%\desc{本项目基于真实数据预测用户是否会为车辆购买保险,使用Python和Weka为工具。抽样解决了数据不平衡问题;使用信息增益率进行特征选择和特征提取;训练了多个模型 (决策树 / 朴素贝叶斯 / 逻辑回归 / SVM)。}
%\sectionsep

\subsection{Design and Implementation of Credit Evaluation System in Commercial Banks}
\subtitle{Undergraduate Thesis \hfill 2014/02 - 2014/05}
\desc{Inspired by the intern experience in Standard Chartered Bank, this financial system tries to automate the credit evaluation process in the commercial banks. The system is constructed using Java EE, and implemented using the industry standard tools (hibernate, struts and spring).}
\sectionsep


\end{minipage} 


\end{document}
\documentclass[]{article}
