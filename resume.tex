% !TEX TS-program = xelatex
% !TEX encoding = UTF-8 Unicode
% !Mode:: "TeX:UTF-8"

\documentclass[12pt,a4paper]{resume}
\begin{document}
\XeTeXlinebreaklocale "zh"

%%%%%%%%%%
%
% HEADER
%
%%%%%%%%%%
\namesection{叶俊杰}{}

\vspace{5pt}
\subtitle{1991年, 男, 现居上海}
\vspace{5pt}
\subtitle{电话: 15867835270}
\vspace{5pt}
\subtitle{邮箱: junjieyemain@outlook.com}

\indent\makebox[\linewidth]{\rule{\paperwidth}{0.4pt}}



%%%%%%%%%%
%
% COLUMN ONE
%
%%%%%%%%%%
\begin{minipage}[t]{0.32\textwidth} 

%%%%%%%%%%
% EDUCATION
\section{教育经历}

\subsection{英国爱丁堡大学}
\subtitle{理学硕士 \hfill 2014年9月 - 2015年11月}
\desc{
	信息学院人工智能专业 \\
	%MSc in Artificial Intelligence \\
	专业绩点: 65 (一等荣誉学位为70)
}
\sectionsep

\subsection{浙江财经大学}
\subtitle{本科双学位 \hfill 2010年9月 - 2014年6月}
\desc{
	信息学院计算机科学与技术专业 \\
	专业绩点: 88/100 \\
	财政学院税务专业 \\
	专业绩点: 89/100
}
\sectionsep

%%%%%%%%%%
% PROFILE
\section{个人主页}
LinkedIn:  \href{https://www.linkedin.com/in/junjie-ye-49477b126/}{\custombold{Junjie Ye}}

\sectionsep

%%%%%%%%%%
% SKILL
\section{技术技能}

\subsection{技术证书}
\vspace{13pt}
%\vspace{\topsep}
\begin{lists} \desc {
	\item 中级软件工程师
}
\end{lists}

\subsection{技术领域}
\vspace{13pt}
\begin{lists} \desc{
	\item 机器学习 / 深度学习
	\item 自然语言处理 / 语音识别
	\item 知识图谱
	\item 大数据处理
	\item 网络爬虫
}
\end{lists}

\subsection{技术框架}
\vspace{13pt}
\begin{lists} \desc{
	\item Tensorflow / Theano / Keras
	\item Hadoop / Spark / Hbase
	\item 
	\item MongoDB / Neo4j / ArangoDB
	\item Redis
	\item 网络爬虫
}
\end{lists}

\subsection{编程语言}
\vspace{13pt}
\begin{lists} \desc{
	\item Python
	\item Java
	\item Shell
	\item Matlab
	\item C / C++
	\item SQL
	\item JavaScript
	\item HTML / CSS
}
\end{lists}


\subtitle{其他熟悉语言:}
%\vspace{\topsep}
C \textbullet{} C++ \textbullet{} Assembly \textbullet{} \LaTeX \textbullet{} Markdown \\
%\begin{lists}
%\item C, C++, Assembly, \LaTeX, Markdown
%\end{lists}
%\textbullet{} 商业银行信用评估系统 \\
\sectionsep

\subsection{操作平台}
\subtitle{Mac OSX, Linux, Windows}
\sectionsep

\subsection{语言技能}
\subtitle{英语:}
%\vspace{\topsep}
\begin{lists}
\item 雅思IELTS: 7.5 / 9
\item 英文文档写作优秀
\end{lists}
\sectionsep



%%%%%%%%%%
%
% COLUMN TWO
%
%%%%%%%%%%
\end{minipage} 
\hfill
\begin{minipage}[t]{0.63\textwidth} 

\section{工作经历}

\subsection{竹间智能科技(上海)有限公司 \hfill 中国上海}
\subtitle{研发工程师 \hfill 2016年4月 - 至今}
\vspace{12pt}
\begin{lists} \desc{
\item 协助评估企业信用: 转换,输入并检查企业财务报表至银行内部信用评估系统。跟踪报告客户信用评估报告。
\item 协助维护银行金融系统。}
\end{lists}
\sectionsep

\subsection{渣打银行 \hfill 中国宁波}
\subtitle{信息与网络安全部门实习生 \hfill 2013年10月 - 2014年1月}
\begin{lists}
\item 协助评估企业信用: 转换,输入并检查企业财务报表至银行内部信用评估系统。跟踪报告客户信用评估报告。
\item 协助维护银行金融系统。
\end{lists}
\sectionsep


\subsection{宁波市信息产业局 \hfill 中国宁波}
\subtitle{网络与通信管理处实习生 \hfill 2013年1月 - 2013年2月}
\begin{lists}
\item 学习电子政务,电子商务,企业信息化的相关知识。学习网络信息安全技术及设备和产品的监督管理内容。了解IT产业的相关技术政策,技术体制和技术标准等。
\end{lists}
\sectionsep

\subsection{施乐公司 \hfill 中国宁波}
\subtitle{技术部和调度中心实习生 \hfill 2012年7月 - 2012年8月,}
\begin{lists}
\item 在技术部学习计算机维护和软件故障处理。
\item 在调度中心学习产品和人员调度管理。
\end{lists}
\sectionsep

\subsection{中国电信 \hfill 中国宁波}
\subtitle{网络运行维护事业部实习生 \hfill 2011年7月 - 2011年8月,}
\begin{lists}
\item 学习局域网组建,网络参数配置,ISP与局域网连接配置,以及相关故障处理等。
\end{lists}
\sectionsep


%%%%%%%%%%
% RESEARCH
\section{研究经历}

\subsection{递归神经网络声学模型}
\subtitle{硕士毕业论文项目 \hfill 2015年6月 - 2015年8月}
\desc{本项目使用递归神经网络构建自动语音识别系统中的声学模型。项目目标是使用递归神经网络声学模型获得比深度神经网络声学模型更优的语音识别表现,并且探索进一步提高语音识别系统表现的其他方法。本项目使用Theano构建递归神经网络模型,使用Kaldi构建语音识别系统的其他部分。本项目成功证明了项目假设,即使用递归神经网络构建声学模型,能够使语音识别系统获得更优的总体表现。项目也探索了影响递归神经网络表现的其他因素。}
\sectionsep

\subsection{商业银行信用评估系统的设计与实现}
\subtitle{本科毕业论文项目 \hfill 2014年2月 - 2014年5月}
\desc{本项目是受渣打银行实习经验启发,诣在于实现商业银行信用评估系统的流程自动化。一般商业银行的信用评估系统大都需要人工逐个输入和检查数据,常有易出错,效率低下,界面复杂难用等问题。项目采用Java EE技术构建金融系统,使用业界通用的框架工具(Hibernate,Struts,Spring)进行开发。项目论文从需求分析入手,依照软件工程原则,逐步设计和实现子系统。项目系统具备一般信用评估系统的基本功能,并加入了自动上传报表文件,自动信用评估(初级)等新功能。}
\sectionsep


\end{minipage} 


\end{document}
\documentclass[]{article}
